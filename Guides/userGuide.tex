\documentclass[letter]{article}

\usepackage[english]{babel}
\usepackage[utf8]{inputenc}

\title{Train Management System User Guide}

\author{Benjamin Weichman, Caroline Quinn, and Alex Keeling}

\date{}

\begin{document}
\maketitle

\section{How to Navigate the Application}
From the main webpage, select the link for your role. Choices are: 
\begin{itemize}
\item Passenger
\item Travel Agent
\item Employee
\item Manager (Employee Scheduler)
\item Manager (Train Scheduler)
\item Head Office
\item System Admin
\end{itemize}

If you select the wrong view, simply click your browser's `Go Back' button, and select the correct view. \\

When you're finished, simply close your browser window or tab.

% End Navigation Section
%%%%%%%%%%%%%%%%%%%%%%%%%%%%%%%%%%%%%%%%%%%%%%%%%%%%%%%%%%%%%%%

\section{How to use the Application}

\subsection{Passenger}
\subsubsection{How to use}
How to view all train voyages- 
After selecting the Passenger view  
1. Click the “view all voyages” button
All future voyages will be listed on the page 

How to view personal train voyage information-
After selecting the Passenger view  
1.Type you full name into the dialogue box
2. Click “View Your Tickets” 
All past, current and future voyages you own tickets for will be displayed. 


\subsubsection{Limitations \& Abilities}
Use this view if you a registered passenger in the train database (own an account), of if you would like to view upcoming train voyages. This view will allow you to display all of your personal train ticket information (past and future).  This view will also allow you to see all upcoming train voyages.

\subsection{Travel Agent}
\subsubsection{How to use}

\subsubsection{Limitations \& Abilities}
Use this view if you work for an independent, mega, regional, of consortium travel agency. This view will allow you to view up to date vacancies in different classes on train voyages, (including the voyages departure, place, time and date, and arrival place time and date).  You will be able to add your customers to the train database, and purchase ,and cancel tickets on their behalf. This view will also allow you delete, update and create customer information in the train database. You will not be able to update customer ticket information, this must be done by deleting and repurchasing a ticket. 


\subsection{Employee}
\subsubsection{How to use}

\subsubsection{Limitations \& Abilities}
Use this view if you work for the train company. The view will list all train voyages you will be scheduled to work on.  This view will not allow you to trade/drop/add shifts to your schedule, you with talk to your scheduling manager  to do so.


\subsection{Manager (Employee Scheduler)}
\subsubsection{How to use}

\subsubsection{Limitations \& Abilities}
Use this view to schedule staff for train voyage. This view will allow you to view all train voyages and the train and cars (including engine) involved in the voyages in order to appropriately schedule staff. You will also be able to add, delete and update employee information.


\subsection{Manager (Train Scheduler)}
\subsubsection{How to use}

\subsubsection{Limitations \& Abilities}
Use this view to schedule train voyages. This view will allow you to add, update and delete train voyages that will occur in the future. You cannot add new train routes, all voyages must run on existing routes. 


\subsection{Head Office}
\subsubsection{How to use}

\subsubsection{Limitations \& Abilities}
Use this view allows you to change rudimentary aspects of the train system .This view allows you to add, update, and delete train routes, track sections, train stations, and engines.


\subsection{System Admin}
\subsubsection{How to use}
This is just like entering a query at a command prompt. Standard MySQL queries will work here. See the Programmer's Guide and Design Document for details on table names and fields.

\subsubsection{Limitations \& Abilities}
This allows you to enter any free form query you like. Note that this view should only be used by someone with a knowledge of MySQL. Improper use of this view could easily delete all data in your system, and/or break your system. Please be careful.


%%%%%%%%%%%%%%%%%%%%%%%%%%%%%%%%%%%%%%%%%%%%%%%%%%%%%%%%%%%%
% end 'how to use it' section

\end{document}
