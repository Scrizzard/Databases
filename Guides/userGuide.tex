\section{How to Navigate the Application}
From the main webpage, select the link for your role. Choices are: 
\begin{itemize}
\item Passenger
\item Travel Agent
\item Employee
\item Manager (Employee Scheduler)
\item Manager (Train Scheduler)
\item Head Office
\item System Admin
\end{itemize}

If you select the wrong view, simply click your browser's `Go Back' button, and select the correct view. \\

When you're finished, simply close your browser window or tab.

% End Navigation Section
%%%%%%%%%%%%%%%%%%%%%%%%%%%%%%%%%%%%%%%%%%%%%%%%%%%%%%%%%%%%%%%

\section{How to use the Application}

\subsection{Passenger}
\subsubsection{How to use}

\subsubsection{Limitations \& Abilities}



\subsection{Travel Agent}
\subsubsection{How to use}

\subsubsection{Limitations \& Abilities}



\subsection{Employee}
\subsubsection{How to use}

\subsubsection{Limitations \& Abilities}



\subsection{Manager (Employee Scheduler)}
\subsubsection{How to use}

\subsubsection{Limitations \& Abilities}



\subsection{Manager (Train Scheduler)}
\subsubsection{How to use}

\subsubsection{Limitations \& Abilities}



\subsection{Head Office}
\subsubsection{How to use}

\subsubsection{Limitations \& Abilities}



\subsection{System Admin}
\subsubsection{How to use}
This is essentially like entering a query at a command prompt. Standard MySQL queries will work here. See the Programmer's Guide and Design Document for details on table names and fields.

\subsubsection{Limitations \& Abilities}
This allows you to enter any free form query you like. Note that this view should only be used by someone with a knowledge of MySQL. Improper use of this view could easily delete all data in your system, and/or break your system. Please be careful.


%%%%%%%%%%%%%%%%%%%%%%%%%%%%%%%%%%%%%%%%%%%%%%%%%%%%%%%%%%%%
% end 'how to use it' section