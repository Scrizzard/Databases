\documentclass[letter]{article}

\usepackage[english]{babel}
\usepackage[utf8]{inputenc}

\title{Train Management System User Guide}

\author{Benjamin Weichman, Caroline Quinn, and Alex Keeling}

\date{}

\begin{document}
\maketitle

\section{How to Navigate the Application}
From the main webpage, select the link for your role. Choices are: 
\begin{itemize}
\item Passenger
\item Travel Agent
\item Employee
\item Manager (Employee Scheduler)
\item Manager (Train Scheduler)
\item Head Office
\item System Admin
\end{itemize}

If you select the wrong view, simply click your browser's `Go Back' button, and select the correct view. \\

When you're finished, simply close your browser window or tab.

% End Navigation Section
%%%%%%%%%%%%%%%%%%%%%%%%%%%%%%%%%%%%%%%%%%%%%%%%%%%%%%%%%%%%%%%

\section{How to use the Application}

\subsection{Passenger}
\subsubsection{How to use}

How to view all train voyages:\\ 
After selecting the Passenger view \\\
1. Click the ``view all voyages" button\\
All future voyages will be listed on the page\\ 

How to view personal train voyage information:\\
After selecting the Passenger view;  \\
1.Type you full name into the dialogue box.\\
2. Click ``View Your Tickets" button.\\
All past, current and future voyages you own tickets for will be displayed. 


\subsubsection{Limitations \& Abilities}
Use this view if you a registered passenger in the train database (own an account), of if you would like to view upcoming train voyages. This view will allow you to display all of your personal train ticket information (past and future).  This view will also allow you to see all upcoming train voyages.

\subsection{Travel Agent}
\subsubsection{How to use}

How to add customer to database:\\
After selecting the travelAgent view\\
1.Click the “Passenger Help” button\\
2. Click the “Add new Passenger” button. \\
3. Type all applicable, correct information into textboxes.\\
4. Click the “Submit” button.\\
A message will appear indicating the success of your passenger creation.\\


How to update customer in database:\\
After selecting the travelAgent view\\
1.Click the “Passenger Help” button.\\
2. Click the “Update Passenger” button\\
3. Type the existing passengers passengerID into the textbox.\\
4.Click the “Submit” button. \\
4. 3. Type all applicable, correct new information into textboxes.\\
All past, current and future voyages you own tickets for will be displayed. \\
A message will appear indicating the success of your update to passenger information.\\ 


\subsubsection{Limitations \& Abilities}
Use this view if you work for an independent, mega, regional, of consortium travel agency. This view will allow you to view up to date vacancies in different classes on train voyages, (including the voyages departure, place, time and date, and arrival place time and date).  You will be able to add your customers to the train database, and purchase ,and cancel tickets on their behalf. This view will also allow you delete, update and create customer information in the train database. You will not be able to update customer ticket information, this must be done by deleting and repurchasing a ticket. 


\subsection{Employee}
\subsubsection{How to use}
How to view employee schedule:\\
After selecting the Employee view;\\ 
1.Type your employeeID into the text box.\\
2. Click the ``Submit" button. \\
Your will be redirected to a new page containing all the information about your shifts. \\


\subsubsection{Limitations \& Abilities}
Use this view if you work for the train company. The view will list all train voyages you will be scheduled to work on.  This view will not allow you to trade/drop/add shifts to your schedule, you with talk to your scheduling manager  to do so.


\subsection{Manager (Employee Scheduler)}
\subsubsection{How to use}
To view the staff schedule for a particular voyage, enter the voyage departure date and time, and click the `View Schedule' button. \\
To add an employee to voyage once the schedule is displayed, enter the employee's name and click the `Add to Voyage' button. \\
To view an employee's employment information, enter their name in the employee search box, and click submit. This will display an employee's information. To delete said employee, click the `Delete' button. \\
To add a new Employee, click the `Add New Employee' button. Enter the name, title, and starting year. Click the submit button to finish adding the new employee.

\subsubsection{Limitations \& Abilities}
Use this view to schedule staff for train voyage. This view will allow you to view all train voyages and the train and cars (including engine) involved in the voyages in order to appropriately schedule staff. You will also be able to add, delete and update employee information.


\subsection{Manager (Train Scheduler)}
\subsubsection{How to use}

\subsubsection{Limitations \& Abilities}
Use this view to schedule train voyages. This view will allow you to add, update and delete train voyages that will occur in the future. You cannot add new train routes, all voyages must run on existing routes. 


\subsection{Head Office}
\subsubsection{How to use}
The `View Current Routes', `View Current Track Sections', `View Current Stations', and `View Current Engine' buttons allow you to examine the current state of your system.
To add a train route, click the `Add Route' button. Select the number of stations on the new route, and enter the route's distance, base ticket cost, and travel time. Clicking the submit button will allow you to choose the sections of track for your new route, and their order. Once you've selected all the stations on your route, click submit again, and your new route will be added to the system. To verify its existence, you can view the current routes.\\
To add a track section, click the `Add Track Section' button. Select the origin and terminal stations for this section of track, and set the initial `In Service' state. Click submit, to finalize this section of track. \\
To add a station, click the `Add Station' button. Enter a name for the new station, the latitude, longitude, and address. Click submit to finish adding the new station. \\
To add an engine, click the `Add Engine' button. Select the type of engine, enter the year of construction, and select an initial `In Service' state. Click submit and your new engine will be added.
\subsubsection{Limitations \& Abilities}
Use this view allows you to change rudimentary aspects of the train system.This view allows you to add, update, and delete train routes, track sections, train stations, and engines.


\subsection{System Admin}
\subsubsection{How to use}
This is just like entering a query at a command prompt. Standard MySQL queries will work here. See the Programmer's Guide and Design Document for details on table names and fields.

\subsubsection{Limitations \& Abilities}
This allows you to enter any free form query you like. Note that this view should only be used by someone with a knowledge of MySQL. Improper use of this view could easily delete all data in your system, and/or break your system. Please be careful.


%%%%%%%%%%%%%%%%%%%%%%%%%%%%%%%%%%%%%%%%%%%%%%%%%%%%%%%%%%%%
% end 'how to use it' section

\end{document}
