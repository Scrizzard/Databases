\documentclass[letter]{article}

\usepackage[english]{babel}
\usepackage[utf8]{inputenc}

\title{Train Management System User Guide}

\author{Benjamin Weichman, Caroline Quinn, and Alex Keeling}

\date{}

\begin{document}
\maketitle

\section{How to Navigate the Application}
From the main webpage (URL), select the link for your role. Choices are: 
\begin{itemize}
\item Passenger
\item Travel Agent
\item Employee
\item Manager (Employee Scheduler)
\item Manager (Train Scheduler)
\item Head Office
\item System Admin
\end{itemize}

If you select the wrong view, simply click your browser's `Go Back' button, and select the correct view. \\

When you are finished, simply close your browser window or tab.

% End Navigation Section
%%%%%%%%%%%%%%%%%%%%%%%%%%%%%%%%%%%%%%%%%%%%%%%%%%%%%%%%%%%%%%%

\section{How to use the Application}

\subsection{Passenger}
\subsubsection{How to use}

How to view all train voyages:\\ 
After selecting the Passenger view,\\\
1. Click the “view all voyages” button.\\
All future voyages will be listed on the page.\\ 

How to view all voyages for which a given user owns tickets:\\
After selecting the Passenger view, \\
1.Select your name from the drop down menu "please select you name".\\
2. Click the “View Your Voyages” button.\\
All voyages for which you own tickets will be displayed. 


\subsubsection{Limitations \& Abilities}
Use this view if you are a registered passenger in the train database (eg, you own an account), or if you would like to view upcoming train voyages. This view will allow you to display all of your personal train ticket information.

\subsection{Ticket Agent}
\subsubsection{How to use}

How to view voyage availability information:\\
After selecting the Ticket Agent view,\\  
1. Scroll down to the "View Voyages" header.
2. Click on “View Voyages” . \\
3.Select the number of voyages you would like to view per page from the drop down menu.\\
4.Navigate through pages of voyage information by clicking "Previous" or "Next" at the bottom of the voyage table.\\
NOTE: By Clicking  voyage information headers (ex:Route ID) the voyage information will be reordered in preference to this sections ordering.\\ 
All future voyages and their information will appear in a table below.\\ 


How to add a passenger to database:\\
After selecting the Ticket Agent view:\\
1. Scroll down to the "Create Passenger" header.
2. Click the “Create Passenger” .\\
3. Enter all requested passenger information into the text boxes in this (Create Passenger) section.\\
4. Click the “Submit” button.\\
A new passenger has now been added to the database. 


How to remove a passenger from database:\\
After selecting the Ticket Agent view,\\
1. Scroll down to the "View and Remove Passenger" header.
2. Click the “View and Remove Passenger”.\\
3.Select the number of passengers you would like to view per page from the drop down menu.\\
4. Navigate through pages of passenger information by clicking "Previous" or "Next" at the bottom of the passenger table.\\
5. Select the passenger you would like to remove. \\
NOTE:You may search for a passenger by entering their name/passenger ID or any other information in the search box located above the passenger table.
NOTE:This option also allows you to view all passenger in the database.\\


How to purchase ticket on behalf of a passenger:\\
After navigating to the Ticket Agent view,\\ 
1. Scroll down to the "Create Ticket" header.
2. Click the “Create Ticket”.\\
2. Select the passenger you would like to purchase a ticket for from the first drop down menu.\\ 
3. Select the voyage the passenger would like to purchase a ticket for from the second down menu.\\
4. Select the train car the passenger would like to be one from the third drop down menu.\\ 
5. Click the "submit" button.\\ 


How to cancel a ticket on behalf of passenger:\\
After selecting the Ticket Agent view,\\
1. Scroll down to the "View and Remove Tickets" header.
2. Click the “View and Remove Tickets”.\\
3.Select the number of Tickets you would like to view per page from the drop down menu.\\
4. Navigate through pages of ticket information by clicking "Previous" or "Next" at the bottom of the ticket table.\\
5. Select the passenger you would like to remove. \\
NOTE:You may search for a ticket by entering any applicable information (Ticket ID, Passenger ID ect.) in the search box located above the ticket table.
NOTE:This option also allows you to view all tickets in the database.\\




\subsubsection{Limitations \& Abilities}
Use this view if you work for an independent, mega, regional, or consortium travel agency. This view will allow you to see up-to-date vacancies on train voyages in addition to their departure time, arrival time, origin, and destination. You will be able to add your customers to the train database and purchase and cancel tickets on their behalf. This view will also allow you delete, update and create passengers. You will not be able to update customer ticket information. This may only be accomplished through deleting and repurchasing a ticket. 


\subsection{Employee}
\subsubsection{How to use}
How to view employee schedule:\\
After navigating to the Employee view,\\ 
1. Type an employee ID into the text box.\\
2. Click the "Submit" button. \\
You will be redirected to a new page containing all the information about the chosen employee's shifts.\\


\subsubsection{Limitations \& Abilities}
Use this view if you work for the train company. The view will list all train voyages you will be scheduled to work on.  This view will not allow you to add or remove scheduled shifts. These privileges are reserved for the Employee Scheduler view.


\subsection{Employee Manager}
\subsubsection{How to use}


\subsubsection{Limitations \& Abilities}
Use this view to schedule staff for train voyage. This view will allow you to see all train voyages and their associated engines and cars in order to make informed scheduling decisions. You will also be able to add, delete and update employee information.


\subsection{Manager (Train Scheduler)}
\subsubsection{How to use}
How To Schedule a new voyage:\\
After navigating to the Train Scheduler view,\\ 
1. Select a route you would like the voyage to follow from the "Route" drop down menu.\\
2. Select an engine you would like to pull the train on its voyage from the "Engine" drop down menu.\\
3. Select the departure day of the voyage from the "Day" drop down menu.\\
4. Select the departure month of the voyage from the "Month" drop down menu.\\
5. Select the departure year of the voyage from the "Year" drop down menu.\\
6. Select the departure Hour of the voyage from the "Hour" drop down menu.\\
7. Select the departure Minutes of the voyage from the "Minutes" drop down menu.\\
8. Click the "submit" button. 
NOTE: You may view all voyages including recently added ones in the table below. You may select the number of voyages you would like to view per page from the drop down menu, and navigate between pages with the "previous" and "next" buttons located at the bottom of the table. 


\subsubsection{Limitations \& Abilities}
Use this view to schedule train voyages. This view will allow you to add, update and delete future train voyages. You cannot add new train routes. All voyages must run on existing routes.


\subsection{Head Office}
\subsubsection{How to use}
The `View Current Routes', `View Current Track Sections', `View Current Stations', and `View Current Engines' buttons allow you to examine the current state of your system.
To add a train route, click the `Add Route' button. Select the number of stations on the new route, and enter the route's distance, base ticket cost, and travel time. Clicking the submit button will allow you to choose the sections of track for your new route, and their order. Once you've selected all the stations on your route, click submit again, and your new route will be added to the system. To verify its existence, you can view the current routes.\\
To add a track section, click the `Add Track Section' button. Select the origin and terminal stations for this section of track, and set the initial `In Service' state. Click submit, to finalize this section of track. \\
To add a station, click the `Add Station' button. Enter a name for the new station, the latitude, longitude, and address. Click submit to finish adding the new station. \\
To add an engine, click the `Add Engine' button. Select the type of engine, enter the year of construction, and select an initial `In Service' state. Click submit and your new engine will be added.
\subsubsection{Limitations \& Abilities}
Use this view allows you to change rudimentary aspects of the train system.This view allows you to add, update, and delete train routes, track sections, train stations, and engines.

\subsection{Database Admin}
\subsubsection{How to use}
This is akin to entering query in a command prompt. Free-form MySQL queries may be used here. See the Relational Diagram or Design Document for descriptions of table names and fields.

\subsubsection{Limitations \& Abilities}
This allows you to enter any free form query you like. This view should only be used by someone with a proper knowledge of MySQL. Improper use of this view could result in destruction of data and the removal of integrity constraints. Please be careful.


%%%%%%%%%%%%%%%%%%%%%%%%%%%%%%%%%%%%%%%%%%%%%%%%%%%%%%%%%%%%
% end 'how to use it' section

\end{document}
